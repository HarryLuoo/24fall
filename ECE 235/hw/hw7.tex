
% --------------------------------------------------------------
% This part is the preamble, which you don't have to worry about
% To type your solutions, scroll down to where it says "Start here"
% --------------------------------------------------------------

% This defines the formatting for the document
\documentclass[12pt]{article}
\usepackage[left=1in,top=1in,right=1in,bottom=1in]{geometry}
\usepackage{enumitem}
\setlist{noitemsep}
\setlist[enumerate,1]{label=(\alph*)}
\setlist[enumerate,2]{label=(\roman*)}

% This imports the packages which we will need to type math symbols
\usepackage{amsmath,amssymb}

% This defines the "problem" and "solution" environments
\usepackage{amsthm}
\theoremstyle{definition}\newtheorem{problem}{Problem}
\newenvironment{solution}{\begin{proof}[\bfseries\textup{Solution:}]}{\end{proof}}

\begin{document}

% --------------------------------------------------------------
%                         Start here
% --------------------------------------------------------------

% This is the title:
\begin{center}
\bfseries ECE 235 % Fill in the course name and section here
\
Homework 7 % Fill in the homework number here
\
Harry Luo % Fill in your name here
\end{center}

% Problem 1 -----------------------------------------------------
\begin{problem}
% Type or paste the problem statement here.
\end{problem}
\begin{solution}
    \subsection*{part a}

    The drift current density in a semiconductor is given by the sum of contributions from electrons and holes:
    $$
    J_{\text{drift}} = e p \mu_p E + e n \mu_n E
    $$
    where:
    \begin{itemize}
        \item \( p \) and \( n \) are the hole and electron concentrations (\( \text{cm}^{-3} \)),
        \item \( \mu_p \) and \( \mu_n \) are the hole and electron mobilities (\( \text{cm}^2/\text{Vs} \))
    \end{itemize}
    In an n-type semiconductor, where \( n \gg p \), the hole contribution \( e p \mu_p E \) becomes negligible, simplifying the expression to:
    $$
    J_{\text{drift}} \approx e n \mu_n E
    $$
    
    \subsection*{b}
    
    Given:
    \begin{itemize}
        \item Donor concentration \( N_D = 10^{16} \, \text{cm}^{-3} \) (thus \( n \approx N_D = 10^{16} \, \text{cm}^{-3} \)),
        \item Electron mobility \( \mu_n = 1350 \, \text{cm}^2/\text{Vs} \),
        \item Electric field \( E = 100 \, \text{V/cm} \).
    \end{itemize}
    
    $$
    J_{\text{drift}} = e n \mu_n E = (-1.6 \times 10^{-19} \, \text{C}) \times (10^{16} \, \text{cm}^{-3}) \times (1350 \, \text{cm}^2/\text{Vs}) \times (100 \, \text{V/cm})
    $$
    Algebra yields:
    $$
    J_{\text{drift}} = - 1.6 \times 10^{-19} \times 10^{16} \times 1350 \times 100 = - 216 \, \text{A/cm}^2
    $$
    
    \subsection*{c}
    With increased doping:
    \begin{itemize}
        \item Donor concentration \( N_D = 10^{18} \, \text{cm}^{-3} \),
        \item Reduced electron mobility \( \mu_n = 400 \, \text{cm}^2/\text{Vs} \).
    \end{itemize}
    
    $$
    J_{\text{drift}} = e n \mu_n E = (-1.6 \times 10^{-19} \, \text{C}) \times (10^{18} \, \text{cm}^{-3}) \times (400 \, \text{cm}^2/\text{Vs}) \times (100 \, \text{V/cm})
    $$
    Algebra yields:
    $$
    J_{\text{drift}} = -1.6 \times 10^{-19} \times 10^{18} \times 400 \times 100 = -6400 \, \text{A/cm}^2
    $$
    
    Increasing the donor concentration from \( 10^{16} \, \text{cm}^{-3} \) to \( 10^{18} \, \text{cm}^{-3} \) results in a significant increase in electron concentration (\( n \)) by a factor of 100. Although the electron mobility (\( \mu_n \)) decreases due to enhanced scattering from the higher dopant density, the overall drift current density \( J_{\text{drift}} \) increases substantially because the rise in carrier concentration outweighs the reduction in mobility.
\end{solution}

% Problem 2 -----------------------------------------------------
\newpage
\begin{problem}

\end{problem}
\begin{solution}


    \subsection*{Part (a) }
    The diffusion current density is given by Fick's first law:
    $$J_p = -qD_p\frac{dp}{dx}$$
    
    Given $\Delta x = 50\,\mu\text{m} = 5\times10^{-3}\,\text{cm}$, the concentration gradient is:
    $$\frac{dp}{dx} = \frac{5\times10^{16} - 10^{17}}{5\times10^{-3}} = -10^{19}\,\text{cm}^{-4}$$
    
    Therefore:
    $$J_p = -(1.6\times10^{-19})(12)(-10^{19}) = 19.2\,\text{A}/\text{cm}^2$$
    
    \subsection*{Part (b) }
    The drift current density is:
    $$J_{drift} = qp\mu_pE$$
    
    Using average hole concentration $\bar{p} = \frac{10^{17} + 5\times10^{16}}{2} = 7.5\times10^{16}\,\text{cm}^{-3}$:
    $$J_{drift} = (1.6\times10^{-19})(7.5\times10^{16})(450)(10^3) = 5400\,\text{A}/\text{cm}^2$$
    
    The drift current density is significantly larger than the diffusion current density due to the strong electric field of $10^3\,\text{V}/\text{cm}$, suggesting that drift mechanisms dominate over diffusion in carrier transport within the p-type silicon sample..

\end{solution}

% Continue adding more problems as needed...

\newpage
\begin{problem}

\end{problem}
\begin{solution}
    \subsection*{(a)}

For electrons in semiconductors:
$$
J_{\text{total}} = J_{\text{drift}} + J_{\text{diff}} = q n(x) \mu_n E - q D_n \frac{dn(x)}{dx}
$$
where $q = 1.6 \times 10^{-19}$ C (electron charge)

\subsection*{(b) Calculations at $x = 10 \mu m$ }

1. Electron concentration at $x = 10 \mu m$:
$$
n(x) = 5 \times 10^{15} + (3 \times 10^{14})(10^{-3}) = 5.0003 \times 10^{15} \text{ cm}^{-3}
$$

2. Drift current density:
\begin{align*}
J_{\text{drift}} &= q n \mu_n E \\
&= (1.6 \times 10^{-19})(5.0003 \times 10^{15})(1350)(50) \\
&= 54.0032 \text{ A/cm}^2
\end{align*}

3. Diffusion current density:
\begin{align*}
J_{\text{diff}} &= q_e D_n \frac{dn}{dx} \\
&= (-1.6 \times 10^{-19})(35)(3 \times 10^{14}) \\
&= -1.68 \times 10^{-3} \text{ A/cm}^2
\end{align*}

4. Total current density:
$$
J_{\text{total}} = 54.0032 - 0.00168 \approx 54 \text{ A/cm}^2
$$


\end{solution}
The drift current is the primary contributor to the total current in this silicon sample under the given conditions.
The diffusion current is present but contributes minimally due to its much smaller magnitude.
\end{document}
\documentclass[12pt]{article}
\usepackage{amsmath}
\usepackage{amsfonts}
\usepackage{amssymb}
\usepackage{geometry}
\usepackage{graphicx}
\usepackage{hyperref}

\geometry{a4paper, margin=1in}

\title{Electric Potential of an Ideal Dipole in a Spherical Dielectric Shell}
\author{}
\date{}

\begin{document}

\maketitle

\section*{Problem Statement}

Consider an ideal electric dipole \(\mathbf{p} = p\,\hat{z}\) located at the center of a spherical dielectric shell. The shell has an inner radius \(a\) and an outer radius \(b\), and it is characterized by a uniform dielectric constant \(\epsilon_r\). Our goal is to determine the electric potential \(V(r, \theta)\) in all three regions:

\begin{itemize}
    \item \textbf{Region I:} Inside the cavity (\(r < a\))
    \item \textbf{Region II:} Within the dielectric shell (\(a < r < b\))
    \item \textbf{Region III:} Outside the dielectric shell (\(r > b\))
\end{itemize}

We will solve this problem using two different approaches: the Separation of Variables Method and the Successive Approximations Method.

\section*{Solution Using Separation of Variables Method}

\subsection*{Region I: Inside the Cavity (\(r < a\))}

In the cavity, the potential is solely due to the dipole at the center. The potential of an ideal dipole in free space is given by:
\[
V_I(r, \theta) = \frac{\mathbf{p} \cdot \hat{\mathbf{r}}}{4\pi \epsilon_0 r^2} = \frac{p \cos \theta}{4\pi \epsilon_0 r^2}
\]

\subsection*{Region II: Within the Dielectric Shell (\(a < r < b\))}

Within the dielectric shell, the potential satisfies Laplace's equation:
\[
\nabla^2 V = 0
\]
Due to spherical symmetry and the absence of free charges in the dielectric, the general solution can be expressed as:
\[
V_{II}(r, \theta) = \sum_{l=0}^{\infty} \left( A_l r^l + \frac{B_l}{r^{l+1}} \right) P_l(\cos \theta)
\]
For a dipole (\(l = 1\)), the dominant term is \(l = 1\):
\[
V_{II}(r, \theta) = \left( A_1 r + \frac{B_1}{r^2} \right) P_1(\cos \theta) = \left( A_1 r + \frac{B_1}{r^2} \right) \cos \theta
\]

\subsection*{Region III: Outside the Dielectric Shell (\(r > b\))}

Outside the dielectric shell, the potential is influenced by both the original dipole and the polarization charges on the dielectric boundary. The general solution in this region is:
\[
V_{III}(r, \theta) = \sum_{l=0}^{\infty} \frac{C_l}{r^{l+1}} P_l(\cos \theta)
\]
Again, focusing on the dipole term (\(l = 1\)):
\[
V_{III}(r, \theta) = \frac{C_1}{r^2} \cos \theta
\]

\subsection*{Boundary Conditions at \(r = a\)}

1. \textbf{Continuity of Potential:}
\[
V_I(a, \theta) = V_{II}(a, \theta) \Rightarrow \frac{p \cos \theta}{4\pi \epsilon_0 a^2} = \left( A_1 a + \frac{B_1}{a^2} \right) \cos \theta
\]
Simplifying:
\[
\frac{p}{4\pi \epsilon_0 a^2} = A_1 a + \frac{B_1}{a^2} \quad \text{(1)}
\]

2. \textbf{Continuity of Electric Displacement Field:}
The radial component of the electric field is:
\[
E_{r, \text{I}} = -\frac{\partial V_I}{\partial r} = \frac{2p \cos \theta}{4\pi \epsilon_0 r^3}
\]
\[
E_{r, \text{II}} = -\frac{\partial V_{II}}{\partial r} = -\left( A_1 - \frac{2B_1}{r^3} \right) \cos \theta
\]
Applying the boundary condition:
\[
\epsilon_0 E_{r, \text{I}}(a) = \epsilon_r \epsilon_0 E_{r, \text{II}}(a) \Rightarrow \epsilon_0 \left( \frac{2p}{4\pi \epsilon_0 a^3} \cos \theta \right) = \epsilon_r \epsilon_0 \left( -A_1 + \frac{2B_1}{a^3} \right) \cos \theta
\]
Simplifying:
\[
\frac{p}{2\pi a^3} = -\epsilon_r \left( A_1 - \frac{2B_1}{a^3} \right) \quad \text{(2)}
\]

\subsection*{Boundary Conditions at \(r = b\)}

1. \textbf{Continuity of Potential:}
\[
V_{II}(b, \theta) = V_{III}(b, \theta) \Rightarrow \left( A_1 b + \frac{B_1}{b^2} \right) \cos \theta = \frac{C_1}{b^2} \cos \theta
\]
Simplifying:
\[
A_1 b + \frac{B_1}{b^2} = \frac{C_1}{b^2} \quad \text{(3)}
\]

2. \textbf{Continuity of Electric Displacement Field:}
\[
E_{r, \text{II}} = -\left( A_1 - \frac{2B_1}{r^3} \right) \cos \theta
\]
\[
E_{r, \text{III}} = -\frac{\partial V_{III}}{\partial r} = \frac{2C_1}{r^3} \cos \theta
\]
Applying the boundary condition:
\[
\epsilon_r \epsilon_0 E_{r, \text{II}}(b) = \epsilon_0 E_{r, \text{III}}(b) \Rightarrow \epsilon_r \epsilon_0 \left( -A_1 + \frac{2B_1}{b^3} \right) \cos \theta = \epsilon_0 \left( \frac{2C_1}{b^3} \cos \theta \right)
\]
Simplifying:
\[
-\epsilon_r \left( A_1 - \frac{2B_1}{b^3} \right) = \frac{2C_1}{b^3} \quad \text{(4)}
\]

\subsection*{Solving the System of Equations}

We have four equations: (1), (2), (3), and (4). Solving these equations, we find the coefficients \(A_1\), \(B_1\), and \(C_1\).

\[
A_1 = \frac{p}{4\pi \epsilon_0 \epsilon_r a^3} \cdot \frac{1}{b + a \epsilon_r}
\]
\[
B_1 = \frac{p a^3}{4\pi \epsilon_0 \epsilon_r} \cdot \frac{1}{b^3 + a^3 \epsilon_r}
\]
\[
C_1 = \frac{p b^3}{4\pi \epsilon_0 \epsilon_r (b + a \epsilon_r)}
\]

\subsection*{Final Expressions for the Potential}

\textbf{Region I (\(r < a\)):}
\[
V_I(r, \theta) = \frac{p \cos \theta}{4\pi \epsilon_0 r^2}
\]

\textbf{Region II (\(a < r < b\)):}
\[
V_{II}(r, \theta) = \left( \frac{p}{4\pi \epsilon_0 \epsilon_r a^3 (b + a \epsilon_r)} r + \frac{p a^3}{4\pi \epsilon_0 \epsilon_r r^2 (b^3 + a^3 \epsilon_r)} \right) \cos \theta
\]

\textbf{Region III (\(r > b\)):}
\[
V_{III}(r, \theta) = \frac{p b^3}{4\pi \epsilon_0 \epsilon_r (b + a \epsilon_r) r^2} \cos \theta
\]

\section*{Solution Using Successive Approximations Method}

\subsection*{Initial Potential in Free Space}

The dipole potential in free space is:
\[
V_0(r,\theta) = \frac{p\cos\theta}{4\pi\epsilon_0r^2}
\]

\subsection*{Effect of Inner Boundary (r=a)}

At \(r=a\), we have a boundary between free space and dielectric. From Problem 4.38, we know the form of solution:

For \(r < a\):
\[
V_I(r,\theta) = \frac{p\cos\theta}{4\pi\epsilon_0r^2}
\]

For \(a < r < b\):
\[
V_{II}(r,\theta) = \frac{p\cos\theta}{4\pi\epsilon_0r^2}\left[A + B\left(\frac{r^3}{a^3}\right)\right]
\]
where \(A\) and \(B\) are constants to be determined.

\subsection*{Boundary Conditions at \(r=a\)}

1. \textbf{Continuity of Potential:}
\[
\left.V_I\right|_{r=a} = \left.V_{II}\right|_{r=a} \Rightarrow \frac{p\cos\theta}{4\pi\epsilon_0a^2} = \frac{p\cos\theta}{4\pi\epsilon_0a^2}\left[A + B\right]
\]
Simplifying:
\[
A + B = 1
\]

2. \textbf{Continuity of Electric Displacement Field:}
\[
\epsilon_0\left.\frac{\partial V_I}{\partial r}\right|_{r=a} = \epsilon_r\epsilon_0\left.\frac{\partial V_{II}}{\partial r}\right|_{r=a}
\]
\[
\epsilon_0 \left( \frac{2p}{4\pi \epsilon_0 a^3} \cos \theta \right) = \epsilon_r \epsilon_0 \left( -A_1 + \frac{2B_1}{a^3} \right) \cos \theta
\]
Simplifying:
\[
\frac{p}{2\pi a^3} = -\epsilon_r \left( A_1 - \frac{2B_1}{a^3} \right)
\]
Solving these equations:
\[
A = \frac{\epsilon_r + 2}{3}
\]
\[
B = \frac{1-\epsilon_r}{3}
\]

\subsection*{Effect of Outer Boundary (r=b)}

Now we need to consider the outer boundary. The potential will have the form:

For \(a < r < b\):
\[
V_{II}(r,\theta) = \frac{p\cos\theta}{4\pi\epsilon_0r^2}\left[\frac{\epsilon_r + 2}{3} + \frac{1-\epsilon_r}{3}\left(\frac{r^3}{a^3}\right)\right]
\]

For \(r > b\):
\[
V_{III}(r,\theta) = \frac{p\cos\theta}{4\pi\epsilon_0r^2}C
\]
where \(C\) is a constant to be determined.

\subsection*{Boundary Conditions at \(r=b\)}

1. \textbf{Continuity of Potential:}
\[
V_{II}(b, \theta) = V_{III}(b, \theta) \Rightarrow \frac{p\cos\theta}{4\pi\epsilon_0b^2}\left[\frac{\epsilon_r + 2}{3} + \frac{1-\epsilon_r}{3}\left(\frac{b^3}{a^3}\right)\right] = \frac{p\cos\theta}{4\pi\epsilon_0b^2}C
\]
Simplifying:
\[
C = \frac{\epsilon_r + 2}{3} + \frac{1-\epsilon_r}{3}\left(\frac{b^3}{a^3}\right)
\]

\subsection*{Final Expressions for the Potential}

\textbf{Region I (\(r < a\)):}
\[
V_I(r,\theta) = \frac{p\cos\theta}{4\pi\epsilon_0r^2}
\]

\textbf{Region II (\(a < r < b\)):}
\[
V_{II}(r,\theta) = \frac{p\cos\theta}{4\pi\epsilon_0r^2}\left[\frac{\epsilon_r + 2}{3} + \frac{1-\epsilon_r}{3}\left(\frac{r^3}{a^3}\right)\right]
\]

\textbf{Region III (\(r > b\)):}
\[
V_{III}(r,\theta) = \frac{p\cos\theta}{4\pi\epsilon_0r^2}\left[\frac{\epsilon_r + 2}{3} + \frac{1-\epsilon_r}{3}\left(\frac{b^3}{a^3}\right)\right]
\]

\section*{Conclusion}

Both the Separation of Variables Method and the Successive Approximations Method provide rigorous and mathematically sound solutions to the problem of an ideal dipole within a spherical dielectric shell. Upon careful analysis and appropriate limiting processes, both methods yield results that are consistent with the standard textbook expressions for the electric potential outside and inside the dielectric sphere.

\end{document}
% --------------------------------------------------------------
% This part is the preamble, which you don't have to worry about
% To type your solutions, scroll down to where it says "Start here"
% --------------------------------------------------------------

% This defines the formatting for the document
\documentclass[12pt]{article}
\usepackage[left=1in,top=1in,right=1in,bottom=1in]{geometry}
\usepackage{enumitem}
\setlist{noitemsep}
\setlist[enumerate,1]{label=(\alph*)}
\setlist[enumerate,2]{label=(\roman*)}

% This imports the packages which we will need to type math symbols
\usepackage{amsmath,amssymb,graphicx}

% This defines the "problem" and "solution" environments
\usepackage{amsthm}
\theoremstyle{definition}\newtheorem{problem}{Problem}
\newenvironment{solution}{\begin{proof}[\bfseries\textup{Solution:}]}{\end{proof}}

\begin{document}


% --------------------------------------------------------------
%                         Start here
% --------------------------------------------------------------

% This is the title:
\begin{center}
\bfseries Math 421, Section 1 
\\ 
Homework 1
\\ 
Harry Luo % Fill in your name here
\\ [24pt] 
\end{center}

%
% Problem 1 -----------------------------------------------------
\begin{problem}[De Morgan's laws]
Let A and B be statements.  Use a truth table to prove the following:
\begin{enumerate}
\item ``Not (A and B)'' is equivalent to ``(not A) or (not B)''.
\item ``Not (A or B)'' is equivalent to ``(not A) and (not B)''.
\end{enumerate}
\end{problem}
\begin{solution} \quad \newline

(a): \begin{center} % This puts the following in the middle of the page
    \begin{tabular}{ c|c|c|c|c|c} % This starts the table
    % (and says that we want 3 centered columns separaterd by vertical lines)
    A & B & not (A and B) & not A & not B& (not A) or (not B)\\ % This is the first row of entries.  
    % (Use "&" to seperate entries, and "\\" to end the line.)
    \hline % This creates a horizontal line
    T & T & F&F&F&F \\ % Second row of entries...
    T & F &T &F &T &T \\ 
    F & T &T &T &F & T\\ 
    F & F &T &T &T &T\\
    \end{tabular}
    \end{center}
We have shown that the columns for ``not (A and B)'' and ``(not A) or (not B)'' are the same, so the two statements are equivalent. 
\newline
    
(b): \begin{center}
    \begin{tabular}{ c|c|c|c|c|c}
    A & B & not (A or B) & not A & not B& (not A) and (not B)\\
    \hline
    T & T & F&F&F&F \\
    T & F &F &F &T &F \\
    F & T &F &T &F & F\\
    F & F &T &T &T &T\\

        
    \end{tabular}
\end{center}
We have shown that the columns for ``not (A or B)'' and ``(not A) and (not B)'' are the same, so the two statements are equivalent.

\end{solution}



%
% Problem 2 -----------------------------------------------------
\newpage
\begin{problem}[The distributive property]
Let A, B, and C be statements.  Use a truth table to prove the following:
\begin{enumerate}
\item ``A and (B or C)'' is equivalent to ``(A and B) or (A and C)''.
\item ``A or (B and C)'' is equivalent to ``(A or B) and (A or C)''.
\end{enumerate}
\end{problem}
\begin{solution}
\quad \newline
    
    (a): \begin{center}
    \begin{tabular}{ c|c|c|c|c|c|c|c}
    A & B & C & A and (B or C) & A and B & A and C & (A and B) or (A and C)\\
    \hline
    T & T & T & T & T & T & T\\
    T & T & F & T & T & T & T\\
    T & F & T & T & F & T & T\\
    T & F & F & F & F & F & F\\
    F & T & T & F & F & F & F\\
    F & T & F & F & F & F & F\\
    F & F & T & F & F & F & F\\
    F & F & F & F & F & F & F\\
    \end{tabular}
\end{center}
We have shown that the columns for ``A and (B or C)'' and ``(A and B) or (A and C)'' are the same, so the two statements are equivalent.\newline

(b):

\begin{center}

\begin{tabular}{ c|c|c|c|c|c|c|c}
A & B & C & A or (B and C) & A or B & A or C & (A or B) and (A or C)\\
\hline
T & T & T & T & T & T & T\\
T & T & F & T & T & T & T\\
T & F & T & T & T & T & T\\
T & F & F & T & T & T & T\\
F & T & T & T & T & T & T\\
F & T & F & F & T & F & F\\
F & F & T & F & F & T & F\\
F & F & F & F & F & F & F\\
\end{tabular}

\end{center}

We have shown that the columns for ``A or (B and C)'' and ``(A or B) and (A or C)'' are the same, so the two statements are equivalent.

\end{solution}

%
% Problem 3 -----------------------------------------------------
\newpage
\begin{problem}
Let A and B be statements.  If we know that A implies B, which one of the following can we conclude?
\begin{enumerate}
\item A cannot be false.
\item A and B are both true.
\item If A is false, then B is false.
\item B cannot be false.
\item If B is false, then A is false.
\item If B is true, then A is true.
\item At least one of A and B is true.
\end{enumerate}
\end{problem}
\begin{solution} \quad \newline

    (e) is the correct conclusion. For an implication, the only way for it to be true while its consequent is false, is to construct a false antecedent. Therefore, if B is false, then A must be false.

\end{solution}

%
% Problem 4 -----------------------------------------------------
\newpage
\begin{problem}
Negate the following sentences:
\begin{enumerate}
\item If there is a job worth doing, then it is worth doing well.
\item Every cloud has a silver lining.
\item For every complex problem, there is an answer that is clear, simple, and wrong.
\end{enumerate}
\end{problem}
\begin{solution}\quad \newline

    (a): We denote: $ A $ as "there is a job worth doing" and $ B $ as "it is worth doing well". The original sentence can be written as $ A \implies B $. The negation of this sentence is:  
    In English this is 
    \fbox{"There is a job worth doing, and it is not worth doing well."}
\\

    (b): The negation of a universal statement is to find an exsistential counterexample. Thus the negation of "Every cloud has a silver lining" is \fbox{There is a cloud without a silver lining.}\\

    (c): We denote: \begin{itemize}
        \item  $x$ = a complex problem.
        \item $ X $ = set of all complex problems.
        \item $y$ = an answer.
        \item $ Y $ = set of all answers.
        \item $ C(y) $= $ y $ is clear. 
        \item $S(y)$ = $y$ is simple.
        \item $W(y)$ = $y$ is wrong.
    \end{itemize}
    
The statement can be translated as: 
    $$\forall x \in X , \; \exists y \in Y \, \text{s.t.} (C(y) \, \text{and} S(y) \, \text{and} \, W(y)))$$

Its negation is:
\begin{align}
&\exists x \in X , \; \forall y \in Y \, \text{s.t.}  (\text{not}\{ C(y) \, \text{and} S(y) \, \text{and} \, W(y)\}) \\
= & \exists x \in X , \; \forall y \in Y \, \text{s.t.}  ((\text{not}\, C(y)) \, \text{and} (\text{ not}\, S(y)) \, \text{and} \, (\text{ not} W(y)))
\end{align}
 In English, this reads: 
 
 \fbox{\parbox{0.8\linewidth}{"There is at least one complex problem that doesn't have any answer that is simultaneously clear, simple, and wrong."}}


\end{solution}

% Problem 5 -----------------------------------------------------
\newpage
\begin{problem}
Let A, B, and C be statements.  Negate the following sentences:
\begin{enumerate}
\item At least one of A and B are true.
\item Both A and B are false.
\item At least two of A, B, and C are false.
\end{enumerate}
\end{problem}
\begin{solution}  \, \\
    
    
    (a) \begin{align}
        &\text{translation: } &&A \text{ or } B \\
        &\text{negation: } &&\text{not} (A \text{ or } B) = \text{not} A \text{ and } \text{not} B
    \end{align}

    (b) \begin{align}
        &\text{translation: } &&\text{not} A \text{ and } \text{not} B \\
        &\text{negation: } &&\text{not} (\text{not} A \text{ and } \text{not} B) = A \text{ or } B
    \end{align}

    (c) \begin{align}
        \text{  translation: } &(\text{not}\,A \text{ and}\,\text{not}\,B) \text{ or}\, (\text{not}\,A \text{ and}\, \text{ not}\,C) \text{ or}\,(\text{ not}\,B \text{ and}\,\text{ not}\,C)\\
        \text{negation: } &\text{not}((\text{not}\,A \text{ and}\,\text{not}\,B) \text{ or}\, (\text{not}\,A \text{ and}\, \text{ not}\,C) \text{ or}\,(\text{ not}\,B \text{ and}\,\text{ not}\,C))\\
        & = ( A \text{ or}\, B) \text{ and}\, (A  \text{ or}\, C) \text{ and}\, (B \text{ or}\, C)
    \end{align}
In English, the negation is \fbox{at least two of A, B, and C are true.}
\end{solution}

% Problem 6 -----------------------------------------------------
\newpage
\begin{problem}
Let $X$ be a set, and let $P(x)$ be a statement about elements $x$ in $X$.  Negate the following sentences:
\begin{enumerate}
\item For every $x$ in $X$, there is a $y$ in $X$ not equal to $x$, for which $P(y)$ is true.
\item If $P(x)$ and $P(y)$ are both true, then $x = y$.
\end{enumerate}
\end{problem}
\begin{solution} \, \\

(a)translation: $$ 
    \forall x \in X, \exists y \in X \, \text{ s.t.}\, ( y \neq x\, \text{and}\, P(y)) 
$$ 

Negation: $$ 
     \exists x \in X, \forall y \in X \, \text{ s.t.}\, ( y = x\, \text{or}\, \text{not} P(y))
$$ 


(b) translation:$$ 
    \forall x, y \in X , \text{ s.t.}\,( (P(x) \text{ and }\, P(y)) \implies (x = y))
$$ 

Negation:
\begin{align}
 &   \exists x, y \in X \; \text{ s.t. }\,\text{ not}\, (\text{ not}\,(P(x) \text{  and}\, P(y)) \text{ or}\, (x = y))\\  
=  &\exists x, y \in X \; \text{ s.t.}\, (  P(x) \text{ and}\,  P(y) \text{ and}\, (x \neq y))\,
\end{align}

\end{solution}

\end{document}
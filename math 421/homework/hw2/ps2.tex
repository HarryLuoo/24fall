% --------------------------------------------------------------
% This part is the preamble, which you don't have to worry about
% To type your solutions, scroll down to where it says "Start here"
% --------------------------------------------------------------

% This defines the formatting for the document
\documentclass[12pt]{article}
\usepackage[left=1in,top=1in,right=1in,bottom=1in]{geometry}
\usepackage{enumitem}
\setlist{noitemsep}
\setlist[enumerate,1]{label=(\alph*)}
\setlist[enumerate,2]{label=(\roman*)}

% This imports the packages which we will need to type math symbols
\usepackage{amsmath,amssymb}

% This defines the "problem" and "solution" environments
\usepackage{amsthm}
\theoremstyle{definition}\newtheorem{problem}{Problem}
\newenvironment{solution}{\begin{proof}[\bfseries\textup{Solution:}]}{\end{proof}}

\usepackage{parskip} 
\usepackage{hyperref}

\begin{document}

% --------------------------------------------------------------
%                         Start here
% --------------------------------------------------------------

% This is the title:
\begin{center}
\bfseries Math 421, Section 1 
\\ 
Homework 2
\\ 
Harry Luo % Fill in your name here
\\ [24pt] 
\end{center}

% Problem 1 -----------------------------------------------------
\begin{problem}
Prove that for any $x,y\in\mathbb{N}$, if $x$ is odd and $y$ is odd then $x+y$ is even.
\end{problem}


\begin{solution} 
    Suppose $ x, y \in \mathbb{N} $ are odd, then $ \exists n, m \in \mathbb{N} \cup \{0\} \ \; s.t. \; x = 2n+1, y = 2m+1. $ 
    \begin{equation}
         x+y = 2 n + 1 + 2 m + 1 = 2(n + m + 1). 
    \end{equation}  
    Since $ (n + m + 1) \in \mathbb{N}, x+y $ is even.  
\end{solution}

\begin{equation}
    \iiint_{\text{Cube}} \left( -e^{-x} - e^{-y} - e^{-z} \right) , dV = l^2 \left[ \left( e^{-\left( x_0 + \frac{l}{2} \right)} - e^{-\left( x_0 - \frac{l}{2} \right)} \right) + \left( e^{-\left( y_0 + \frac{l}{2} \right)} - e^{-\left( y_0 - \frac{l}{2} \right)} \right) + \left( e^{-\left( z_0 + \frac{l}{2} \right)} - e^{-\left( z_0 - \frac{l}{2} \right)} \right) \right]. 
\end{equation}
% Problem 2 -----------------------------------------------------
\newpage
\begin{problem}
Prove that for any $x\in\mathbb{N}$, if $x$ is odd then $x^3$ is odd.
\end{problem}

\begin{solution}


Suppose $ x  $ is odd, i.e. $ \exists n \in \mathbb{N} \cup \{0\} \, s.t.\, x=2n+1$ 

\begin{equation} x^3 = (2n+1)^3 = 8n^3 + 12n^2 + 6n + 1 = 2(4n^3 + 6n^2 + 3n) + 1. \end{equation} 

    Trivially, since $ (4n^3 + 6n^2 + 3n) \in \mathbb{N} \cup \{0\}, x^3 $ is odd.

\end{solution} 



% Problem 3 -----------------------------------------------------
\newpage
\begin{problem}
Using induction, prove that for all $n\in\mathbb{N}$ we have
\begin{equation*}
1+3+5+\dots+(2n-1) = n^2 .
\end{equation*}
\end{problem}

\begin{solution}

[Base case]: For $ n = 1 $, we have $ 1 =1 $. Upon careful reflection, this holds true. 


[Inductive step]: Suppose the statement is true for $ \exists n \in \mathbb{N}  $ , i.e. 
\begin{equation} 1+ 3+ \dots + 2n-1 = n^2 \end{equation}
 Then for $ n = n+1 $  we have: 
\begin{align}
    1+3+\dots + 2n-1 + 2(n+1)-1 &= n^2 + 2n+1 \\
    &= (n+1)^2
\end{align}
So the formula is true for $ n+1 $. Thus, by induction, the statement is true for all $ n \in \mathbb{N} $.

\end{solution}

% Problem 4 -----------------------------------------------------
\newpage
\begin{problem}
Compute the following sum:
\begin{equation*}
\frac{1}{1\cdot 3} + \frac{1}{3\cdot 5} + \dots + \frac{1}{(2n-1)(2n+1)} .
\end{equation*}
Prove that your answer is true for all $n\in\mathbb{N}$ using induction.
\end{problem}


\begin{solution}
By noticing $\frac{1}{(2n-1)(2n+1)} = \frac{1}{2} (\frac{1}{2n-1} - \frac{1}{2n+1}) $, a rough calculation suggests that the sum should be $ \frac{1}{2} - \frac{1}{4n+2} $. It is proved by induction as follows: 

[base case]: For $ n=1 $, we have $$ \frac{1}{1\cdot 3} = \frac{1}{2} - \frac{1}{6} = \frac{1}{3}, $$ which is true. 

[Inductive case]: Suppose the statement is true for $ \exists n \in \mathbb{N} $, i.e. \begin{align} 
\frac{1}{1 \cdot 3} + \frac{1}{3 \cdot 5} + \dots + \frac{1}{(2n-1)(2n+1)} &= \frac{1}{2} - \frac{1}{4n+2}
\end{align}
 Then for $ n = n+1, $ \begin{align} 
\frac{1}{1 \cdot 3} &+ \frac{1}{3 \cdot 5} + \dots + \frac{1}{(2n-1)(2n+1)} + \frac{1}{(2n+1)(2n+3)} = \frac{1}{2} - \frac{1}{4n+2} + \frac{1}{(2n+1)(2n+3)}  \\ 
&= \frac{1}{2} - \frac{1}{4n+2} + \frac{1}{2} (\frac{1}{2n+1} - \frac{1}{2n+3}) \\
&= \frac{1}{2} - \frac{1}{4n + 6} \\&= \frac{1}{2} - \frac{1}{4(n+1) + 2} 
 \end{align}
  
So the formula is true for $ n+1 $. Thus, by induction, the statement is true for all $ n \in \mathbb{N} $.





\end{solution}

% Problem 5 -----------------------------------------------------
\newpage
\begin{problem}
Prove the following statements for all $a,b\in\mathbb{R}$:
\begin{enumerate}
\item $-a + (-b) = -(a+b)$.
\item If $a,b\neq 0$ then $a^{-1}\cdot b^{-1} = (ab)^{-1}$.
\end{enumerate}
Carefully justify every step using properties of $\mathbb{R}$ stated in lecture.
\end{problem}

\begin{solution}
[a]:
Consider the original equation, \begin{align} 
    -a + (-b) = -(a+b) 
\end{align}
Adding (a+b) to both sides, we can find that it is equivalent to \begin{align} 
    -a +(-b) + (a+b) = -(a+b)+(a+b) 
\end{align}
Applying inverse addition to the right side, and apply associativity to the left, this is equivalent to 
\begin{align} \label{eq:1}
     -a +a +(-b) +b = 0
\end{align}
Therefore, to prove the original statement, it is suffice to prove its equivalence, i.e. \autoref{eq:1}.
By the inverse addition property, we have \begin{align} 
    -a + a = 0 , \; -b + b = 0 \\
    \Rightarrow -a + a + (-b) + b = 0
\end{align}
The statement is thus proved.

[b]:
Suppose $ a, b \neq 0 $, we have\begin{align} 
    &a^{-1} \cdot b^{-1} \cdot ab \stackrel{\text{commutivity}}{=} a^{-1} \cdot a \cdot b^{-1} \cdot b \stackrel{\text{inverse}}{=}1 \\
 \text{also,} \; &(ab)^{-1} \cdot ab \stackrel{inverse}{=} 1
\end{align}
By transivity, we have \begin{align} 
    a^{-1} \cdot b^{-1} \cdot ab &= (ab)^{-1} \cdot ab \\
    \stackrel{\text{prop.1}}{\Rightarrow} a^{-1} \cdot b^{-1} &= (ab)^{-1}
\end{align}
As desired.
\end{solution}

% Problem 6 -----------------------------------------------------
\newpage
\begin{problem}
Prove the following statements for all $a,b,c,d\in\mathbb{R}$:
\begin{enumerate}
\item If $a<b$ and $c<d$ then $a+c < b+d$.
\item If $0<a<b$ and $0<c<d$ then $ac<bd$.
\end{enumerate}
\end{problem}
\begin{solution}

[a]:
Suppose $ a < b, c < d $, then by O1, \begin{align} 
    a + c < b + c \\
    b + c < d + b .
\end{align}
 By Transitivity, \begin{align} 
     a + c < d + b 
 \end{align}
 By commutivity, \begin{align} 
     a + c < b + d
    \end{align}
Thus proves the inequality.

[b]: Suppose $ 0 < a < b, \, 0<c<d $. Then by O2, \begin{align} 
    a \cdot c < b \cdot c \\
    b \cdot c < b \cdot d 
\end{align}
 By transitivity, \begin{align} 
     a  c < b  d
    \end{align}
    Thus proves the inequality.
\end{solution}

\end{document}
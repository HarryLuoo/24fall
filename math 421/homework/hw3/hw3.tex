% --------------------------------------------------------------
% This part is the preamble, which you don't have to worry about
% To type your solutions, scroll down to where it says "Start here"
% --------------------------------------------------------------

% This defines the formatting for the document
\documentclass[12pt]{article}
\usepackage[left=1in,top=1in,right=1in,bottom=1in]{geometry}
\usepackage{enumitem}
\setlist{noitemsep}
\setlist[enumerate,1]{label=(\alph*)}
\setlist[enumerate,2]{label=(\roman*)}

% This imports the packages which we will need to type math symbols
\usepackage{amsmath,amssymb}
\usepackage{parskip} 

% This defines the "problem" and "solution" environments
\usepackage{amsthm}
\theoremstyle{definition}\newtheorem{problem}{Problem}
\newenvironment{solution}{\begin{proof}[\bfseries\textup{Solution:}]}{\end{proof}}

% Tired of typing \mathbb{...} every time?  Here's a shortcut:
\newcommand{\Z}{\mathbb{Z}}
\newcommand{\R}{\mathbb{R}}

\begin{document}

% --------------------------------------------------------------
%                         Start here
% --------------------------------------------------------------

% This is the title:
\begin{center}
\bfseries Math 421, Section 1 
\\ 
Homework 3
\\ 
Harry Luo % Fill in your name here
\\ [24pt] 
\end{center}

% Problem 1 -----------------------------------------------------
\begin{problem}
Determine whether each of the following functions are injective, surjective, and bijective, and prove your answer.
\begin{enumerate}
\item $f:\Z\to\Z$, $f(x) = 2x$.  % Here, "\Z" is a command defined in the preamble, to output "\mathbb{Z}"
\item $g:\R\to\R$, $g(x) = 2x$.  % Same for "\R" here
\end{enumerate}
\end{problem}
\begin{solution} 
\begin{enumerate}
\item Injectivity: Suppose $ \exists x_1, x_2 \in \Z, s.t. f(x_1) = f(x_2)  $ , want to show: $  x_1 = x_2 .$ 
    \begin{align} 
        f(x_1) = f(x_2) \implies 2x_1 = 2x_2 \implies x_1 = x_2. 
    \end{align}
    
    The function is thus injective.\\
    
Surjectivity: Want to show $ \forall y \in \Z, \exists x \in \Z \; s.t. \; f(x) = y . $ Suppose $x, y \in \Z $, and let $ f(x) = y $. i.e., \begin{align} 
    2x = y \implies x = \frac{y}{2} \in \Z. 
\end{align}
However, $ \frac{y}{2} \in \Z $ only if $ y $ is even. So the above is not true for an arbiturary $ y \in \Z $, contradictory to our assumption. Thus, the function is not surjective.
\\

Collecting the above, the function is not bijective.
\\

\item Injectivity: Suppose $ \exists x_1, x_2 \in \R, s.t. g(x_1) = g(x_2)  $ , want to show: $  x_1 = x_2 .$ 
    \begin{align} 
        g(x_1) = g(x_2) \implies 2x_1 = 2x_2 \implies x_1 = x_2. 
    \end{align}
    
    The function is thus injective.\\


Surejectivity: Suppose $ y \in \R $, we want to find $ x \in \R, \; s.t.\; g(x) = y $.

\begin{align} 
    2x = y \implies x = \frac{y}{2} \in \R.
\end{align}
So the function is surjective.\\

Collecting the above, the function $ g(x) $ is bijective.

\end{enumerate} 


\end{solution}

% Problem 2 -----------------------------------------------------
\newpage
\begin{problem}
Let $f:A\to B$ be a function and $A_1,A_2\subseteq A$ and $B_1,B_2\subseteq B$ be subsets.  Prove the following statements:
\begin{enumerate}
\item $f(A_1\cup A_2) = f(A_1) \cup f(A_2)$.
\item $f(A_1\cap A_2) \subseteq f(A_1) \cap f(A_2)$.
\item $f^{-1}(B_1\cup B_2) = f^{-1}(B_1) \cup f^{-1}(B_2)$.
\item $f^{-1}(B_1\cap B_2) = f^{-1}(B_1) \cap f^{-1}(B_2)$.
\end{enumerate}
\end{problem}
\begin{solution}
    \begin{enumerate}
        \item 
        
        \begin{proof}
            \(\subseteq\): Let \( y \in f(A_1 \cup A_2) \).By definition of image, \(\exists x \in A_1 \cup A_2\) s.t. \( f(x) = y \). \\
            Hence, \( x \in A_1 \) or \( x \in A_2 \). Thus, \( y \in f(A_1) \) or \( y \in f(A_2) \), implying \( y \in f(A_1) \cup f(A_2) \).

            Therefore, $ f(A_1 \cup A_2) \subseteq f(A_1) \cup f(A_2)$ 
            \\

            \(\supseteq\): Let \( y \in f(A_1) \cup f(A_2) \). Then \( y \in f(A_1) \) or \( y \in f(A_2) \). \\
            Thus, \(\exists x \in A_1\) or \( x \in A_2 \) s.t. \( f(x) = y \). \\
            Therefore, \( x \in A_1 \cup A_2 \) and \( y = f(x) \in f(A_1 \cup A_2) \).\\
            Thus, \(f(A_1) \cup f(A_2) \subseteq f(A_1 \cup A_2)\).\\

            Hence, \( f(A_1 \cup A_2) = f(A_1) \cup f(A_2) \).
        \end{proof}
        \bigbreak
        
        \item 
        
        \begin{proof}
            Let \( y \in f(A_1 \cap A_2) \). Then \(\exists x \in A_1 \cap A_2\) s.t. \( f(x) = y \). \\
            Since \( x \in A_1 \) and \( x \in A_2 \), \( y \in f(A_1) \) and \( y \in f(A_2) \). Thus, \( y \in f(A_1) \cap f(A_2) \).
            
            Therefore, \( f(A_1 \cap A_2) \subseteq f(A_1) \cap f(A_2) \).
        \end{proof}  
        \bigbreak
        \item 
        
        \begin{proof}
            \(\subseteq\): Let \( x \in f^{-1}(B_1 \cup B_2) \). Then \( f(x) \in B_1 \cup B_2 \), so \( f(x) \in B_1 \) or \( f(x) \in B_2 \). \\
            Hence, \( x \in f^{-1}(B_1) \) or \( x \in f^{-1}(B_2) \), implying \( x \in f^{-1}(B_1) \cup f^{-1}(B_2) \).\\
            Thus $ f^{-1} (B_1 \cup B_2) \subseteq f^{-1}(B_1) \cup f^{-1}(B_2) $ \\

            \(\supseteq\): Let \( x \in f^{-1}(B_1) \cup f^{-1}(B_2) \). \\
            Then \( x \in f^{-1}(B_1) \) or \( x \in f^{-1}(B_2) \), meaning \( f(x) \in B_1 \) or \( f(x) \in B_2 \). \\
            Thus, \( f(x) \in B_1 \cup B_2 \) and \( x \in f^{-1}(B_1 \cup B_2) \).\\
            So $ f^{-1}(B_1) \cup f^{-1}(B_2) \subseteq f^{-1} (B_1 \cup B_2)   $ \\
            Therefore, \( f^{-1}(B_1 \cup B_2) = f^{-1}(B_1) \cup f^{-1}(B_2) \).
        \end{proof}
        
        \bigbreak
        \item 
        
        \begin{proof}
            \(\subseteq\): Let \( x \in f^{-1}(B_1 \cap B_2) \). Then \( f(x) \in B_1 \cap B_2 \), so \( f(x) \in B_1 \) and \( f(x) \in B_2 \). Hence, \( x \in f^{-1}(B_1) \) and \( x \in f^{-1}(B_2) \), implying \( x \in f^{-1}(B_1) \cap f^{-1}(B_2) \).\\
            So, \( f^{-1}(B_1 \cap B_2) \subseteq  f^{-1}(B_1) \cap f^{-1}(B_2) \)\\

            \(\supseteq\): Let \( x \in f^{-1}(B_1) \cap f^{-1}(B_2) \). Then \( f(x) \in B_1 \) and \( f(x) \in B_2 \), so \( f(x) \in B_1 \cap B_2 \). Thus, \( x \in f^{-1}(B_1 \cap B_2) \).\\
            Therefore, \( f^{-1}(B_1) \cap f^{-1}(B_2) \subseteq f^{-1}(B_1 \cap B_2) \).\\
            
            Therefore, \( f^{-1}(B_1 \cap B_2) = f^{-1}(B_1) \cap f^{-1}(B_2) \).
        \end{proof}
    \end{enumerate}
    
\end{solution}

% Problem 3 -----------------------------------------------------
\newpage
\begin{problem}
Let $f:A\to B$ be a function.  Prove that $f$ is injective if and only if $f(A_1\cap A_2) = f(A_1) \cap f(A_2)$ for all subsets $A_1,A_2\subseteq A$.
\end{problem}
\begin{solution}
    We will prove the equivalence by demonstrating both implications.

    \paragraph{1. \( f \) is injective \( \Rightarrow \) \( \forall A_1, A_2 \subseteq A,\, f(A_1 \cap A_2) = f(A_1) \cap f(A_2) \)}
    
    \begin{proof}
    Assume \( f \) is injective.
    
    \paragraph{\( \subseteq \)}
    
    Let \( y \in f(A_1 \cap A_2) \). Then \( \exists x \in A_1 \cap A_2 \) such that \( f(x) = y \). Since \( x \in A_1 \) and \( x \in A_2 \), it follows that \( y \in f(A_1) \) and \( y \in f(A_2) \). Therefore, \( y \in f(A_1) \cap f(A_2) \).
    
    \paragraph{\( \supseteq \)}
    
    Let \( y \in f(A_1) \cap f(A_2) \). Then \( \exists x_1 \in A_1 \) and \( \exists x_2 \in A_2 \) such that \( f(x_1) = y \) and \( f(x_2) = y \). Since \( f \) is injective, \( x_1 = x_2 \). Let \( x = x_1 = x_2 \). Then \( x \in A_1 \cap A_2 \), and hence \( y = f(x) \in f(A_1 \cap A_2) \).
    
    
    
    Thus, \( f(A_1 \cap A_2) = f(A_1) \cap f(A_2) \) when \( f \) is injective.
    \end{proof}
    
    \paragraph{2. \( \forall A_1, A_2 \subseteq A,\, f(A_1 \cap A_2) = f(A_1) \cap f(A_2) \) \( \Rightarrow \) \( f \) is injective}
    
    \begin{proof}
    Assume \( \forall A_1, A_2 \subseteq A,\, f(A_1 \cap A_2) = f(A_1) \cap f(A_2) \). We aim to show that \( f \) is injective.
    
    Suppose, for contradiction, that \( f \) is not injective. Then \( \exists x_1, x_2 \in A \) with \( x_1 \neq x_2 \) and \( f(x_1) = f(x_2) = y \).
    
    Consider the subsets \( A_1 = \{x_1\} \) and \( A_2 = \{x_2\} \).
    
    
    
    Since \( x_1 \neq x_2 \),
    \[
    A_1 \cap A_2 = \emptyset.
    \]
    Thus,
    \[
    f(A_1 \cap A_2) = f(\emptyset) = \emptyset.
    \]
    
    
    \[
    f(A_1) = \{f(x_1)\} = \{y\}, \quad f(A_2) = \{f(x_2)\} = \{y\},
    \]
    so
    \[
    f(A_1) \cap f(A_2) = \{y\} \cap \{y\} = \{y\}.
    \]
    
    We have
    \[
    f(A_1 \cap A_2) = \emptyset \neq \{y\} = f(A_1) \cap f(A_2),
    \]
    which contradicts the assumption.
    Therefore, \( f \) must be injective.
    \end{proof}
    
    


\end{solution}

% Problem 4 -----------------------------------------------------
\newpage
\begin{problem}
Let $f:A\to B$ be a function.  Prove that the following two statements are equivalent:
\begin{enumerate}
\item The function $f$ is surjective.
\item For every set $C$ and for any functions $g : B \to C$ and $h : B \to C$ such that $g \circ f = h \circ f$, we have $g = h$.
\end{enumerate}
\end{problem}
\begin{solution}

    \textbf{(1) implies (2):}

    Assume \( f \) is surjective. Let \( C \) be an arbitrary set, and let \( g, h: B \to C \) satisfy \( g \circ f = h \circ f \).
    
    For any \( b \in B \), since \( f \) is surjective, there exists \( a \in A \) such that \( f(a) = b \). Therefore,
    \[
    g(b) = g(f(a)) = (g \circ f)(a) = (h \circ f)(a) = h(f(a)) = h(b).
    \]
    Hence, \( g = h \).
    
    \textbf{(2) implies (1):}
    
    Assume statement 2 holds. Suppose, for contradiction, that \( f \) is not surjective. Then there exists \( b_0 \in B \) such that \( b_0 \notin \text{Image}(f) \). We will use this element to construct specific functions \( g \) and \( h \) that satisfy the premise of statement 2 but are not equal, leading to a contradiction:
    
    Let \( C = \{0, 1\} \) and define the functions \( g, h: B \to C \) as follows:
    \[
    g(b) = 
    \begin{cases}
    0 & \text{if } b = b_0, \\
    1 & \text{otherwise},
    \end{cases}
    \quad
    h(b) = 1 \text{ for all } b \in B.
    \]
    Since \( b_0 \notin \text{Image}(f) \), for all \( a \in A \), \( g(f(a)) = 1 = h(f(a)) \). Thus, \( g \circ f = h \circ f \). However, \( g \neq h \) because \( g(b_0) = 0 \) while \( h(b_0) = 1 \), which contradicts the uniqueness condition.
    
    Therefore, \( f \) must be surjective.
    


\end{solution}

% Problem 5 -----------------------------------------------------
\newpage
\begin{problem}
Let $A$ be a nonempty set and $f:A\to A$ a function.  We call $f$ an \emph{involution} if $(f\circ f)(a) = a$ for all $a\in A$.  Prove that if $f:A\to A$ is an involution, then $f$ is bijective.  What is the inverse function $f^{-1}$ in terms of $f$?
\end{problem}
\begin{solution}
    \textbf{1. Injectivity}
    
    Assume that for some \( a_1, a_2 \in A \),
    $$
    f(a_1) = f(a_2).
    $$
    Applying \( f \) to both sides of the equation:
    $$
    f(f(a_1)) = f(f(a_2)).
    $$
    Given that \( f \) is an involution:
    $$
    (f \circ f)(a_1) = (f \circ f)(a_2) \\
    a_1 = a_2.
    $$
    Thus, \( f \) is injective.
    
    \textbf{2. Surjectivity}
    
    Take any element \( b \in A \). Since \( f \) is an involution:
    $$
    f(f(b)) = b.
    $$
    Let \( a = f(b) \). Then:
    $$
    f(a) = f(f(b)) = b.
    $$
    Therefore, for every \( b \in A \), there exists an \( a \in A \) (specifically, \( a = f(b) \)) such that \( f(a) = b \). Therefore \( f \) is surjective.
    
    Since \( f \) is both injective and surjective, it is bijective.
    
    \textbf{Inverse Function}
    
    By definition, the inverse function \( f^{-1} \) satisfies:
    $$
    f^{-1}(f(a)) = a \quad \text{and} \quad f(f^{-1}(a)) = a \quad \text{for all} \quad a \in A.
    $$
    Given that \( f \) is an involution:
    $$
    f(f(a)) = a.
    $$
    Comparing the two conditions, we observe that \( f \) itself satisfies the properties required of an inverse function. Therefore:
    $$
    f^{-1} = f.
    $$
    
    
\end{solution}

% Problem 6 -----------------------------------------------------
\newpage
\begin{problem}
Prove or disprove the following statements:
\begin{enumerate}
\item The set $\{ x\in\R : x\geq 2 \}$ is an interval.
\item The set $\{ x\in\R : x\neq 2 \}$ is an interval.
\end{enumerate}
(Hint:  In order to disprove a statement, you must prove that the negation of the statement is true.)
\end{problem}
\begin{solution}

    \paragraph{1. The Set \( \{ x \in \mathbb{R} \mid x \geq 2 \} \) is an Interval}
    
    \textbf{Proof}:
   
    To confirm that \( S = [2, \infty) \) is indeed an interval, we verify the interval definition.
    
    \begin{enumerate}
        \item \textbf{Take any two elements \( a, b \in S \) with \( a < b \)}:
        \[
        a, b \geq 2 \quad \text{and} \quad a < b.
        \]
        
        \item \textbf{Consider any \( c \in \mathbb{R} \) such that \( a < c < b \)}:
        \[
        a < c < b \quad \text{and} \quad a \geq 2 \quad \Rightarrow \quad c > a \geq 2 \quad \Rightarrow \quad c \geq 2.
        \]
        
        \item \textbf{Thus}, \( c \in S \) since \( c \geq 2 \).
    \end{enumerate}
    
    Since every number between any two elements of \( S \) is also contained within \( S \), \( S \) satisfies the definition of an interval.
    
    \textbf{Conclusion}: The set \( \{ x \in \mathbb{R} \mid x \geq 2 \} \) is an interval, specifically the closed and unbounded interval \( [2, \infty) \).
    
    \subsection*{2. The Set \( \{ x \in \mathbb{R} \mid x \neq 2 \} \) is an Interval}
    
    \textbf{Statement}: The set \( T = \{ x \in \mathbb{R} \mid x \neq 2 \} \) is an interval.
    
    \textbf{Proof}:
    
    We aim to determine whether the set \( T = \mathbb{R} \setminus \{2\} \) satisfies the definition of an interval.
    
    \textbf{Assumption for Contradiction}: Suppose \( T \) is an interval.
    
    \textbf{Analysis}:
    
    1. **Structure of \( T \)**:
        \[
        T = (-\infty, 2) \cup (2, \infty)
        \]
        This is the real line with the single point \( x = 2 \) removed.
    
    2. **Interval Properties**:
        - For \( T \) to be an interval, it must be connected; that is, there should be no "gaps" in \( T \).
        - However, \( T \) explicitly excludes the point \( x = 2 \), creating a discontinuity.
    
    3. **Violation of Interval Definition**:
        - Consider two points \( a = 1 \) and \( b = 3 \) in \( T \), with \( a < 2 < b \).
        - According to the interval definition, every \( c \) such that \( a < c < b \) must be in \( T \).
        - Take \( c = 2 \), which satisfies \( a < c < b \), but \( c = 2 \notin T \).
        - This contradicts the requirement that all intermediate points must be included in the interval.
    
    \textbf{Conclusion}: The set \( T = \{ x \in \mathbb{R} \mid x \neq 2 \} \) is not an interval because it fails to include all real numbers between certain pairs of its elements, specifically excluding the point \( x = 2 \).
    
    \textbf{Summary}
    
    \begin{enumerate}
        \item \textbf{True}: The set \( \{ x \in \mathbb{R} \mid x \geq 2 \} \) is an interval, precisely the closed and unbounded interval \( [2, \infty) \).
        \item \textbf{False}: The set \( \{ x \in \mathbb{R} \mid x \neq 2 \} \) is not an interval, as it excludes the point \( x = 2 \), resulting in a disconnected set.
    \end{enumerate}
    
    \textbf{Final Answer}:
    
    \begin{enumerate}
        \item \textbf{True}. The set \( \{ x \in \mathbb{R} \mid x \geq 2 \} \) is the interval \( [2, \infty) \).
        \item \textbf{False}. The set \( \{ x \in \mathbb{R} \mid x \neq 2 \} \) is not an interval.
    \end{enumerate}
    

\end{solution}

\end{document}
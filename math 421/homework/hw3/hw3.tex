% --------------------------------------------------------------
% This part is the preamble, which you don't have to worry about
% To type your solutions, scroll down to where it says "Start here"
% --------------------------------------------------------------

% This defines the formatting for the document
\documentclass[12pt]{article}
\usepackage[left=1in,top=1in,right=1in,bottom=1in]{geometry}
\usepackage{enumitem}
\setlist{noitemsep}
\setlist[enumerate,1]{label=(\alph*)}
\setlist[enumerate,2]{label=(\roman*)}

% This imports the packages which we will need to type math symbols
\usepackage{amsmath,amssymb}

% This defines the "problem" and "solution" environments
\usepackage{amsthm}
\theoremstyle{definition}\newtheorem{problem}{Problem}
\newenvironment{solution}{\begin{proof}[\bfseries\textup{Solution:}]}{\end{proof}}

% Tired of typing \mathbb{...} every time?  Here's a shortcut:
\newcommand{\Z}{\mathbb{Z}}
\newcommand{\R}{\mathbb{R}}

\begin{document}

% --------------------------------------------------------------
%                         Start here
% --------------------------------------------------------------

% This is the title:
\begin{center}
\bfseries Math 421, Section 1 
\\ 
Homework 3
\\ 
(Name) % Fill in your name here
\\ [24pt] 
\end{center}

% Problem 1 -----------------------------------------------------
\begin{problem}
Determine whether each of the following functions are injective, surjective, and bijective, and prove your answer.
\begin{enumerate}
\item $f:\Z\to\Z$, $f(x) = 2x$.  % Here, "\Z" is a command defined in the preamble, to output "\mathbb{Z}"
\item $g:\R\to\R$, $g(x) = 2x$.  % Same for "\R" here
\end{enumerate}
\end{problem}
\begin{solution}
(Type your solution to problem 1 here.)
\end{solution}

% Problem 2 -----------------------------------------------------
\newpage
\begin{problem}
Let $f:A\to B$ be a function and $A_1,A_2\subseteq A$ and $B_1,B_2\subseteq B$ be subsets.  Prove the following statements:
\begin{enumerate}
\item $f(A_1\cup A_2) = f(A_1) \cup f(A_2)$.
\item $f(A_1\cap A_2) \subseteq f(A_1) \cap f(A_2)$.
\item $f^{-1}(B_1\cup B_2) = f^{-1}(B_1) \cup f^{-1}(B_2)$.
\item $f^{-1}(B_1\cap B_2) = f^{-1}(B_1) \cap f^{-1}(B_2)$.
\end{enumerate}
\end{problem}
\begin{solution}
(Type your solution to problem 2 here.)
\end{solution}

% Problem 3 -----------------------------------------------------
\newpage
\begin{problem}
Let $f:A\to B$ be a function.  Prove that $f$ is injective if and only if $f(A_1\cap A_2) = f(A_1) \cap f(A_2)$ for all subsets $A_1,A_2\subseteq A$.
\end{problem}
\begin{solution}
(Type your solution to problem 3 here.)
\end{solution}

% Problem 4 -----------------------------------------------------
\newpage
\begin{problem}
Let $f:A\to B$ be a function.  Prove that the following two statements are equivalent:
\begin{enumerate}
\item The function $f$ is surjective.
\item For every set $C$ and for any functions $g : B \to C$ and $h : B \to C$ such that $g \circ f = h \circ f$, we have $g = h$.
\end{enumerate}
\end{problem}
\begin{solution}
(Type your solution to problem 4 here.)
\end{solution}

% Problem 5 -----------------------------------------------------
\newpage
\begin{problem}
Let $A$ be a nonempty set and $f:A\to A$ a function.  We call $f$ an \emph{involution} if $(f\circ f)(a) = a$ for all $a\in A$.  Prove that if $f:A\to A$ is an involution, then $f$ is bijective.  What is the inverse function $f^{-1}$ in terms of $f$?
\end{problem}
\begin{solution}
(Type your solution to problem 5 here.)
\end{solution}

% Problem 6 -----------------------------------------------------
\newpage
\begin{problem}
Prove or disprove the following statements:
\begin{enumerate}
\item The set $\{ x\in\R : x\geq 2 \}$ is an interval.
\item The set $\{ x\in\R : x\neq 2 \}$ is an interval.
\end{enumerate}
(Hint:  In order to disprove a statement, you must prove that the negation of the statement is true.)
\end{problem}
\begin{solution}
(Type your solution to problem 6 here.)
\end{solution}

\end{document}